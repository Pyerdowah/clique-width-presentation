\documentclass[polish]{beamer}

% wide screen
% \documentclass[aspectratio=169]{beamer}


%%% YOUR PACKAGES HERE %%%
\usepackage{comment}
\usepackage{hyperref}
\usepackage{tikz} 
\usepackage{caption}
\usetikzlibrary{graphs,graphs.standard,automata,shapes,positioning,calc}

% polish language
\usepackage[polish]{babel}
\usepackage{polski}



%%% IMPORT PG PRESENTATION STYLE %%%
% THIS IS GDANSK UNIVERSITY OF TECHNLOGOGY (PG) PRESENTATION TEMPLATE
% Creator: Jan Cychnerski <jan.cychnerski@eti.pg.edu.pl>
% Copyleft 2019


% PG THEME OPTIONS

\usetheme{Boadilla}
\usecolortheme{default}
\usefonttheme{professionalfonts}

% colors

\definecolor{PGBlue}{RGB}{0,56,101}
\definecolor{PGRed}{RGB}{193,10,39}
\definecolor{PGSilver}{RGB}{200,200,200}
\definecolor{PGBlack}{RGB}{0,0,0}

% PGBlue
\setbeamercolor{frametitle}{fg=PGBlue}
\setbeamercolor{normal text}{fg=PGBlue}
\setbeamercolor{structure}{fg=PGBlue}
\setbeamercolor{item}{fg=PGBlue}

% PGRed
\setbeamercolor{alerted text}{fg=PGRed}
\setbeamercolor{item projected}{fg=PGRed}

% white
\setbeamercolor{title}{fg=white}
\setbeamercolor{titlelike}{fg=white}
\setbeamercolor{subtitle}{fg=white}

% enumerate and itemize styles

\setbeamertemplate{itemize item}{\bfseries\color{PGRed}\raise1pt\hbox{\donotcoloroutermaths$\bullet$}}
\setbeamertemplate{itemize subitem}{\color{PGRed}\raise0.5pt\hbox{--}}
\setbeamertemplate{itemize subsubitem}{\color{PGRed}\tiny\raise1.5pt\hbox{\donotcoloroutermaths$\bullet$}}

\setbeamertemplate{enumerate item}{\bfseries\color{PGRed}\insertenumlabel.}
\setbeamertemplate{enumerate subitem}{\color{PGRed}\insertsubenumlabel.}
\setbeamertemplate{enumerate subsubitem}{\color{PGRed}\insertsubsubenumlabel.}
\setbeamertemplate{enumerate mini template}{\insertenumlabel}


% disable navigation

\beamertemplatenavigationsymbolsempty

% additional commands

\newcommand*{\vcenteredhbox}[1]{\begingroup
\setbox0=\hbox{#1}\parbox{\wd0}{\box0}\endgroup}

\graphicspath{{pgbeamer/}}


\usepackage{iflang}
\IfLanguageName{polish}{
\newcommand{\pglogobig}{pg-logo-big-pl}
\newcommand{\pglogosmall}{pg-logo-small-pl}
}{
\newcommand{\pglogobig}{pg-logo-big-en}
\newcommand{\pglogosmall}{pg-logo-small-en}
}


% FRAME TITLE LOGO
\addtobeamertemplate{frametitle}{\vcenteredhbox{\includegraphics[height=8mm]{\pglogosmall}}\bfseries}{}


\newcommand{\pgtitleframe}{{
% PG TITLE PAGE

\setbeamercolor{background canvas}{bg=PGBlue}
\setbeamercolor{title}{fg=white}
\setbeamercolor*{date}{fg=white}
\setbeamercolor*{author}{fg=white}

\setbeamertemplate{footline}{}

\begin{frame}[noframenumbering]
\centering
\vspace{1cm}
\includegraphics[height=3cm]{\pglogobig}
\vspace{5mm}
\maketitle
\end{frame}
}}

\newcommand{\pglastframe}{{
% PG LAST PAGE

\setbeamercolor{background canvas}{bg=PGBlue}
\setbeamercolor{title}{fg=white}
\setbeamercolor*{date}{fg=white}
\setbeamercolor*{author}{fg=white}

\setbeamertemplate{footline}{}

\begin{frame}[noframenumbering]
\centering
\vspace{1cm}
\includegraphics[height=5cm]{\pglogobig}
\end{frame}
}}



%%% YOUR OPTIONS HERE %%%

\title[Clique-Width: Harnessing the Power of Atoms]{Clique-Width: Harnessing the Power of Atoms}
\subtitle{Narzędzie do rozwiązywania wielu problemów jednocześnie?}
\author{Paulina Brzęcka}
\date{\today}

\setbeamercovered{transparent}
\captionsetup{labelformat=empty}

%%% DOCUMENT BEGINS HERE %%%

\begin{document}

\tikzset{vertex/.style={circle, draw=black, inner sep=0pt, minimum size=0.6cm, text width=0.75cm, align=center}}
\tikzset{cop_vertex/.style={vertex, fill=blue!50}}
\tikzset{rob_vertex/.style={vertex, fill=red!50}}
\tikzset{cop_rob_vertex/.style={vertex, fill=green!50}}
\tikzset{edge/.style={black}}

%%% PG TITLE PAGE %%%
\pgtitleframe

%%% YOUR PRESENTATION HERE %%%

\setbeamercovered{invisible}

\begin{frame}{Motywacje do badań}
    Wiele trudnych problemów z grafami można rozwiązać, ograniczając dane wejściowe do jakiejś klasy grafów.
    Dwa główne pytania brzmią: 
    \begin{itemize}
        \item Dla jakich klas grafów problem grafowy jest wykonalny?
        \item Dla jakich klas grafów jest on trudny obliczeniowo?
    \end{itemize}
    Idealnie byłoby, gdybyśmy chcieli odpowiedzieć na te pytania w odniesieniu do dużego zestawu problemów 
    jednocześnie, zamiast rozważać poszczególne problemy jeden po drugim.\\
    Parametry szerokości grafu pomagają w umożliwieniu takich wyników. Klasa grafów ma ograniczoną 
    szerokość, jeśli istnieje stała c taka, że szerokość wszystkich jej elementów wynosi co najwyżej c.
\end{frame}

\begin{frame}{Definicje}
    \begin{proof}[Podgraf indukowany]
        \renewcommand{\qedsymbol}{}
        Jest to graf, którego zbiór wierzchołków jest zawarty (jest podzbiorem) w zbiorze wierzchołków grafu G,
         a zbiór krawędzi składa się ze wszystkich krawędzi grafu G, 
         których końce należą do zbioru wierzchołków podgrafu.
    \end{proof}
    \begin{examples}
        \begin{columns}[t]
            \column{.2\textwidth}
                \centering
                    \begin{tikzpicture}
                        \node[draw, circle] (A) at (0,2) {};
                        \node[draw, circle] (B) at (1,3) {};
                        \node[draw, circle] (C) at (2,2) {};
                        \node[draw, circle] (D) at (1,1) {};
                        \node[draw, circle] (E) at (2,1) {};
                    
                        \draw (A) -- (B);
                        \draw (A) -- (D);
                        \draw (B) -- (C);
                        \draw (C) -- (D);
                        \draw (C) -- (E);
                        \draw (D) -- (E);
                    \end{tikzpicture}
                    \quad
            \column{.2\textwidth}
                \centering
                \begin{tikzpicture}
                    \node[draw, circle] (A) at (0,2) {};
                    \node[draw, fill=red, circle] (B) at (1,3) {};
                    \node[draw, fill=red, circle] (C) at (2,2) {};
                    \node[draw, fill=red, circle] (D) at (1,1) {};
                    \node[draw, fill=red, circle] (E) at (2,1) {};
                
                    \draw (A) -- (B);
                    \draw (A) -- (D);
                    \draw[red, thick] (B) -- (C);
                    \draw[red, thick] (C) -- (D);
                    \draw[red, thick] (C) -- (E);
                    \draw[red, thick] (D) -- (E);
                  \end{tikzpicture}
                  \quad
            \column{.2\textwidth}  
            \centering
            \begin{tikzpicture}
                \node[draw, circle] (A) at (0,2) {};
                \node[draw, fill=red, circle] (B) at (1,3) {};
                \node[draw, fill=red, circle] (C) at (2,2) {};
                \node[draw, fill=red, circle] (D) at (1,1) {};
                \node[draw, fill=red, circle] (E) at (2,1) {};
            
                \draw (A) -- (B);
                \draw (A) -- (D);
                \draw[red, thick] (B) -- (C);
                \draw[red, thick] (C) -- (D);
                \draw[red, thick] (C) -- (E);
                \draw (D) -- (E);
              \end{tikzpicture}
              \quad    
              \column{.2\textwidth}
              \centering
              \begin{tikzpicture}
                \node[draw, circle] (A) at (0,2) {};
                \node[draw, fill=red, circle] (B) at (1,3) {};
                \node[draw, circle] (C) at (2,2) {};
                \node[draw, fill=red, circle] (D) at (1,1) {};
                \node[draw, fill=red, circle] (E) at (2,1) {};
            
                \draw (A) -- (B);
                \draw (A) -- (D);
                \draw (B) -- (C);
                \draw (C) -- (D);
                \draw (C) -- (E);
                \draw[red, thick] (D) -- (E);
              \end{tikzpicture}
        \end{columns}
    
    \end{examples}
\end{frame}

\begin{frame}{Definicje}
    \begin{proof}[Klasa grafów heredytalnych (dziedzicznych)]
        \renewcommand{\qedsymbol}{}
        Klasa jest dziedziczna wtedy i tylko wtedy, gdy może być scharakteryzowana przez unikalny zbiór $H$ minimalnych, zabronionych podgrafów indukowanych.\\
        Graf $G$ jest $H$-wolny jeśli $G$ nie ma podgrafu indukowanego, izomorficznego do $H$.\\
        Graf $G$ jest ($H_1,...,H_p$)-wolny jeśli $G$ nie ma podgrafów indukowanych, izomorficznych do $H_i$ dla każdego $i$.\\
    \end{proof}
    \begin{examples}
        \begin{columns}[t]
            \column{.2\textwidth}
                \centering
                    \begin{tikzpicture}
                        \node[draw, fill=red, circle] (v1) at (0,0) {};
                        \node[draw, fill=red, circle] (v2) at (1,0) {};
                        \node[draw, fill=red, circle] (v3) at (2,0) {};
                        \node[draw, fill=red, circle] (v4) at (3,0) {};
                        
                        % Edges for P_4
                        \draw[red, thick] (v1) -- (v2);
                        \draw[red, thick] (v2) -- (v3);
                        \draw[red, thick] (v3) -- (v4);
                    \end{tikzpicture}
                    \quad
            \column{.2\textwidth}
                \centering
                \begin{tikzpicture}
                    \node[draw, circle] (v1) at (90:1) {};
                    \node[draw, circle] (v2) at (162:1) {};
                    \node[draw, circle] (v3) at (234:1) {};
                    \node[draw, circle] (v4) at (306:1) {};
                    \node[draw, circle] (v5) at (18:1) {};
                    
                    % Edges for C_5
                    \draw (v1) -- (v2);
                    \draw (v2) -- (v3);
                    \draw (v2) -- (v4);
                    \draw (v3) -- (v5);
                    \draw (v3) -- (v4);
                    \draw (v4) -- (v5);
                    \draw (v5) -- (v1);
                  \end{tikzpicture}
                  \quad
            \column{.2\textwidth}  
            \centering
            \begin{tikzpicture}
                \node[draw, fill=red, circle] (v1) at (90:1) {};
                \node[draw, fill=red, circle] (v2) at (162:1) {};
                \node[draw, fill=red, circle] (v3) at (234:1) {};
                \node[draw, circle] (v4) at (306:1) {};
                \node[draw, fill=red, circle] (v5) at (18:1) {};
                
                % Edges for C_5
                \draw[red, thick] (v1) -- (v2);
                \draw[red, thick] (v2) -- (v3);
                \draw (v3) -- (v4);
                \draw (v4) -- (v5);
                \draw[red, thick] (v5) -- (v1);
                \draw (v2) -- (v4);
              \end{tikzpicture}
              \quad    
        \end{columns}
    \end{examples}
\end{frame}

\begin{frame}{Definicje}
    \begin{proof}[Szerokość kliki]
        \renewcommand{\qedsymbol}{}
        Szerokość kliki grafu G, oznaczona jako cw(G), to minimalna liczba etykiet potrzebnych do skonstruowania G przy użyciu czterech następujących operacji:
        \begin{itemize}
            \item utworzenie nowego grafu składającego się z pojedynczego wierzchołka $v$ z etykietą $i$; 
            \item przyjęcie sumy rozłącznej dwóch grafów oznaczonych etykietami $G_1$ i $G_2$;
            \item dodanie krawędzi pomiędzy każdym wierzchołkiem o etykiecie $i$ a każdym wierzchołkiem o etykiecie j $(i \neq j)$;
            \item oznaczenie każdego wierzchołka etykietą $i$ tak, aby miał etykietę $j$.
        \end{itemize}
    \end{proof}
\end{frame}

\begin{frame}{Szerokość kliki}
    \begin{proof}[]
        \renewcommand{\qedsymbol}{}
        Klasa grafów G ma ograniczoną szerokość kliki, jeśli istnieje stała c taka, że $cw(G) \leq c$ dla każdego $G \in G$; w przeciwnym razie szerokość kliki $G$ jest nieograniczona.
    \end{proof}
    \begin{examples}
        \begin{columns}[t]
            \column{.2\textwidth}
                \centering
                    \begin{tikzpicture}
                        \node[draw, circle] (A) at (0,2) {};
                        \node[draw, circle] (B) at (2,3) {};
                        \node[draw, circle] (C) at (4,2) {};
                        \node[draw, circle] (D) at (3,0) {};
                        \node[draw, circle] (E) at (1,0) {};
                      
                        % Edges for K5 (fully connected)
                        \draw[thick] (A) -- (B);
                        \draw[thick] (A) -- (C);
                        \draw[thick] (A) -- (D);
                        \draw[thick] (A) -- (E);
                        \draw[thick] (B) -- (C);
                        \draw[thick] (B) -- (D);
                        \draw[thick] (B) -- (E);
                        \draw[thick] (C) -- (D);
                        \draw[thick] (C) -- (E);
                        \draw[thick] (D) -- (E);
                    \end{tikzpicture}
            \column{.5\textwidth}
                \centering
                    \begin{tikzpicture}
                        \node[draw, circle] (A) at (3,0) {};
                        \node[draw, circle] (B) at (4,0) {};
                        \node[draw, circle] (C) at (5,0) {};
                        \node[draw, circle] (D) at (6,0) {};
                        \node[draw, circle] (E) at (7,0) {};

                        \node[draw, circle] (F) at (3,1) {};
                        \node[draw, circle] (G) at (4,1) {};
                        \node[draw, circle] (H) at (5,1) {};
                        \node[draw, circle] (I) at (6,1) {};
                        \node[draw, circle] (J) at (7,1) {};
                        \node[draw, circle] (K) at (8,1) {};

                        \node[draw, circle] (L) at (4,2) {};
                        \node[draw, circle] (M) at (5,2) {};
                        \node[draw, circle] (N) at (6,2) {};
                        \node[draw, circle] (O) at (7,2) {};
                        \node[draw, circle] (P) at (8,2) {};

                        \draw[thick] (A) -- (B);
                        \draw[thick] (B) -- (C);
                        \draw[thick] (C) -- (D);
                        \draw[thick] (D) -- (E);

                        \draw[thick] (F) -- (G);
                        \draw[thick] (G) -- (H);
                        \draw[thick] (H) -- (I);
                        \draw[thick] (I) -- (J);
                        \draw[thick] (J) -- (K);

                        \draw[thick] (L) -- (M);
                        \draw[thick] (M) -- (N);
                        \draw[thick] (N) -- (O);
                        \draw[thick] (O) -- (P);

                        \draw[thick] (A) -- (F);
                        \draw[thick] (H) -- (C);
                        \draw[thick] (J) -- (E);
                        \draw[thick] (P) -- (K);
                        \draw[thick] (L) -- (G);
                        \draw[thick] (N) -- (I);
                    \end{tikzpicture}
        \end{columns}
    \end{examples}
\end{frame}


\begin{frame}{Definicje}
    \begin{proof}[Atom]
        \renewcommand{\qedsymbol}{}
        Spójny graf, który nie posiada podgrafu rozpinającego będącego kliką.
    \end{proof}
    \begin{examples}
                \centering
                \begin{tikzpicture}
                    % Nodes for K5
                    \node[draw, circle] (A) at (0,2) {};
                    \node[draw, circle] (B) at (2,3) {};
                    \node[draw, circle] (C) at (4,2) {};
                    \node[draw, circle] (D) at (3,0) {};
                    \node[draw, circle] (E) at (1,0) {};
                  
                    % Additional nodes (outside K5)
                    \node[draw, circle] (F) at (5,4) {};  % New node connected to C
                    \node[draw, circle] (G) at (0,-1) {};  % New node connected to E
                    \node[draw, circle] (H) at (4,-1) {};  % New node connected to D
                  
                    % Edges for K5 (fully connected)
                    \draw[thick] (A) -- (B);
                    \draw[thick] (A) -- (C);
                    \draw[thick] (A) -- (D);
                    \draw[thick] (A) -- (E);
                    \draw[thick] (B) -- (C);
                    \draw[thick] (B) -- (D);
                    \draw[thick] (B) -- (E);
                    \draw[thick] (C) -- (D);
                    \draw[thick] (C) -- (E);
                    \draw[thick] (D) -- (E);
                  
                    % Additional edges from K5
                    \draw[thick] (C) -- (F);  % Edge from C to new node
                    \draw[thick] (E) -- (G);  % Edge from E to new node
                    \draw[thick] (D) -- (H);  % Edge from D to new node
                  \end{tikzpicture}
                \begin{tikzpicture}
                    \node[draw, circle] (A) at (0,0) {};
                    \node[draw, circle] (B) at (2,0) {};
                    \node[draw, circle] (C) at (1,2) {};
                    \node[draw, circle] (D) at (3,1) {};
                    \node[draw, circle] (E) at (-1,1) {};

                    % Edges for the K3 clique (A, B, C)
                    \draw[thick] (A) -- (B);
                    \draw[thick] (B) -- (C);
                    \draw[thick] (C) -- (A);

                    % Additional edges (to show that K3 is not a cut-set)
                    \draw[thick] (A) -- (E);  % A to E
                    \draw[thick] (B) -- (D);  % B to D
                    \draw[thick] (C) -- (D);  % C to D
                    \draw[thick] (E) -- (C);  % E to C
                    \draw[thick] (E) -- (D);  % E to C
                  \end{tikzpicture}
    \end{examples}
\end{frame}

\begin{frame}{Dlaczego atomy?}
    Wiele NP-zupełnych problemów grafowych można rozwiązać w czasie wielomianowym na klasach grafów o ograniczonej szerokości kliki.\\
    Część z nich, między innymi:
    \begin{itemize}
        \item Kolorowanie
        \item Minimalne dopełnienie
        \item Maksymalna klika
        \item Maksymalny ważony zbiór niezależny
        \item Maksymalne indukowane dopasowanie
    \end{itemize}    
    są rozwiązywalne w czasie wielomianowym na grafie dziedzicznym klasy $G$ wtedy i tylko wtedy, gdy ma to miejsce w przypadku atomów $G$. 
\end{frame}

\begin{frame}
    \begin{theorem}
        Klasa atomów $(2P_2, \overline{P_2 + P_3})$-wolnych ma ograniczoną szerokość kliki (mając na uwadze że klasa grafów $(2P_2, \overline{P_2 + P_3})$-wolnych ma nieograniczoną szerokość kliki)
        \begin{figure}[h]
            \centering
            \begin{minipage}{0.45\textwidth}
                \centering
                \begin{tikzpicture}
                    % Pierwsza ścieżka P2
                    \node[draw, circle] (a1) at (0, 0) {};
                    \node[draw, circle] (a2) at (0, 1.5) {};
                    \draw (a1) -- (a2);
                    
                    % Druga ścieżka P2
                    \node[draw, circle] (b1) at (1, 0) {};
                    \node[draw, circle] (b2) at (1, 1.5) {};
                    \draw (b1) -- (b2);
                \end{tikzpicture}
                \caption{$2P_2$} % Podpis bez "Rysunek"
            \end{minipage}
            \hfill
            \begin{minipage}{0.45\textwidth}
                \centering
                \begin{tikzpicture}
                    % Pierwsza ścieżka P2
                    \node[draw, circle] (a1) at (0, 0) {};
                    \node[draw, circle] (a2) at (0, 2) {};
                    
                    % Druga ścieżka P2
                    \node[draw, circle] (b1) at (1, 1) {};
                    \node[draw, circle] (b2) at (2, 0) {};
                    \node[draw, circle] (b3) at (2, 2) {};

                    \draw (a1) -- (a2);
                    \draw (a1) -- (b2);
                    \draw (a2) -- (b3);
                    \draw (a1) -- (b1);
                    \draw (b1) -- (b3);
                    \draw (b1) -- (a2);
                    \draw (b2) -- (b3);
                \end{tikzpicture}
                \caption{$\overline{P_2 + P_3}$} % Podpis bez "Rysunek"
            \end{minipage}
        \end{figure}
    \end{theorem}
\end{frame}

\begin{frame}{Z czego będziemy korzystać?}
    Dla stałej $k \geq 0$ i operacji na grafach $\gamma$, klasa grafów $G'$ jest \emph{($k$, $\gamma$)-uzyskana} z klasy grafów $G$, jeśli:
    \begin{itemize}
        \item[(i)] każdy graf w $G'$ został uzyskany z grafu w $G$ poprzez wykonanie $\gamma$ co najwyżej $k$ razy, oraz
        \item[(ii)] dla każdego $G \in G$, istnieje co najmniej jeden graf w $G'$, uzyskany z $G$ poprzez wykonanie $\gamma$ co najwyżej $k$ razy.
    \end{itemize}
    \begin{enumerate}
        \item Usuwanie wierzchołków zachowuje ograniczoność szerokości kliki.
        \item Dopełnienie podgrafu zachowuje ograniczoność szerokości kliki.
        \item Dopełnienie grafu dwudzielnego zachowuje ograniczoność szerokości kliki.
    \end{enumerate}

    \textbf{Lemat.} Grafy łańcuchowe dwudzielne mają szerokość kliki co najwyżej 3.\\
    \textbf{Lemat.} Atomy podzielone są grafami pełnymi i mają szerokość kliki co najwyżej 2.
\end{frame}

\begin{frame}{Schemat działania}
    \begin{proof}
        \renewcommand{\qedsymbol}{}
        Podejście opiera się na trzech następujących twierdzeniach:
        \begin{enumerate}
            \item $(2P_2, \overline{P_2 + P_3})$-wolnych atomów z indukowanym $C_5$ ma ograniczoną szerokość kliki.
            \item $(2P_2, \overline{P_2 + P_3})$-wolnych atomów z indukowanym $C_4$ ma ograniczoną szerokość kliki.
            \item $(C_4, C_5, 2P_2, \overline{P_2 + P_3})$-wolnych atomów ma ograniczoną szerokość kliki.
        \end{enumerate}
        $(C_4, C_5, 2P_2)$-wolne grafy są grafami podzielonymi i na mocy lematu, atomy tej klasy są grafami pełnymi, więc mają szerokość kliki wynoszącą co najwyżej 2.    
    \end{proof}
\end{frame}

\begin{frame}{Schemat działania}
    \begin{proof}
        \renewcommand{\qedsymbol}{}
        \begin{enumerate}
            \item Dzielimy zbiór wierzchołków dowolnego $(2P_2, \overline{P_2 + P_3})$-wolnego atomu $G$ na różne podzbiory w zależności od ich sąsiedztw w indukowanym $C_5$ lub $C_4$. 
            \item Analizujemy właściwości podzbiorów $V(G)$ oraz ich połączenia, a także wykorzystujemy tę wiedzę do zastosowania odpowiednich operacji $\gamma$.
            \item Te operacje zmienią $G$ w graf $G'$ będący rozłącznym grafem kilku mniejszych grafów, które mają ograniczoną szerokość kliki.
        \end{enumerate}

        To podejście działa, ponieważ:
        \begin{itemize}
            \item stosujemy operacje usuwania wierzchołków, dopełnień podgrafów i dopełnień grafów dwudzielnych tylko ograniczoną liczbę razy;
            \item nie wykorzystujemy właściwości bycia atomem ani bycia $(2P2, \overline{P_2 + P_3})$-wolnym, gdy opuszczamy klasę grafów po zastosowaniu powyższych operacji grafowych.
        \end{itemize}
    \end{proof}
\end{frame}


\begin{frame}
    \begin{lemma}
        Klasa atomów $(2P_2, \overline{P_2 + P_3})$-wolnych atomów z indukowanym $C_5$ ma ograniczoną szerokość kliki.
    \end{lemma}
    \begin{proof}
        \renewcommand{\qedsymbol}{}
        Niech $G$ będzie atomem wolnym od $(2P_2, \overline{P_2 + P_3})$, który zawiera indukowany cykl $C = v_1, \dots, v_5$. Wprowadzamy następujące oznaczenia: dla $S \subseteq \{1, 2, 3, 4, 5\}$, niech $V_S$ oznacza zbiór wierzchołków $x \in V(G) \setminus V(C)$, takich że $N(x) \cap V(C) = \{v_i \mid i \in S\}$.
        
        Rozważamy możliwe wartości $S$.
    \end{proof}
\end{frame}

\begin{frame}
    \begin{theorem}
        \renewcommand{\qedsymbol}{}
        $V_i \cup V_{i,i+1} \cup V_{i-1,i,i+1}$ jest puste dla każdego $i$.
    \end{theorem}
    \textbf{Dowód przez sprzeczność:} $x \in V_2 \cup V_{2,3} \cup V_{1,2,3}$
    \begin{figure}[h]
        \centering
        \begin{tikzpicture}[scale=0.5, every node/.style={draw, circle, inner sep=1pt, minimum size=6pt}]
    
            % Initial cycle C_5 - pentagon shape
            \node (v1) at (90:1.5) {$v_1$};
            \node (v2)[fill=red] at (162:1.5) {$v_2$};
            \node (v3) at (234:1.5) {$v_3$};
            \node (v4)[fill=red] at (306:1.5) {$v_4$};
            \node (v5)[fill=red] at (18:1.5) {$v_5$};
            
            % Draw edges for the initial cycle C_5
            \draw (v1) -- (v2);
            \draw (v2) -- (v3);
            \draw (v3) -- (v4);
            \draw[red, very thick] (v4) -- (v5);
            \draw (v5) -- (v1);
        
            % First neighborhood
            \node (n2) [fill=red, ellipse, draw=red, minimum size=10pt, text=white] at ($(v1) + (2, 1)$) {$V_{1,2,3}$};
            % Edges from first neighborhood to initial cycle

            \draw[red, bend left, very thick] (n2) to (v2);
            \draw[black, bend left] (n2) to (v1);
            \draw[black, bend left] (n2) to (v3);
         
        \end{tikzpicture}
    \end{figure}
    \begin{figure}[h]
        \centering
        \begin{tikzpicture}[scale=0.5, every node/.style={draw, circle, inner sep=1pt, minimum size=6pt}]
    
            % Initial cycle C_5 - pentagon shape
            \node (v1) at (90:1.5) {$v_1$};
            \node (v2) at (162:1.5) {$v_2$};
            \node (v3) at (234:1.5) {$v_3$};
            \node (v4) at (306:1.5) {$v_4$};
            \node (v5) at (18:1.5) {$v_5$};
            
            % Draw edges for the initial cycle C_5
            \draw (v1) -- (v2);
            \draw (v2) -- (v3);
            \draw (v3) -- (v4);
            \draw (v4) -- (v5);
            \draw (v5) -- (v1);
        
            % First neighborhood
            \node (n1) [fill=blue, ellipse, draw=blue, minimum size=10pt, text=white] at ($(v1) + (4, 1)$) {$V_{1,2,4}$};
            \node (n2) [fill=blue, ellipse, draw=blue, minimum size=10pt, text=white] at ($(v1) + (2, 1)$) {$V_{2,3,5}$};
            \node (n3) [fill=blue, ellipse, draw=blue, minimum size=10pt, text=white] at ($(v1) + (-2, 1)$) {$V_{1,2,3,4}$};
            \node (n4) [fill=blue, ellipse, draw=blue, minimum size=10pt, text=white] at ($(v1) + (-4, 1)$) {$V_{1,4}$};
            \node (n5) [fill=blue, ellipse, draw=blue, minimum size=10pt, text=white] at ($(v1) + (8, 1)$) {$V_{1,2,3,4,5}$};

            % Edges from first neighborhood to initial cycle
            \draw[blue, bend left] (n5) to (v1);
            \draw[blue, bend left] (n5) to (v2);
            \draw[blue, bend left] (n5) to (v3);
            \draw[blue, bend left] (n5) to (v4);
            \draw[blue, bend left] (n5) to (v5);
            \draw[blue, bend left] (n1) to (v1);
            \draw[blue, bend left] (n1) to (v2);
            \draw[blue, bend left] (n1) to (v4);
            \draw[blue, bend left] (n2) to (v2);
            \draw[blue, bend left] (n2) to (v3);
            \draw[blue, bend left] (n2) to (v5);
            \draw[blue, bend right] (n3) to (v1);
            \draw[blue, bend right] (n3) to (v2);
            \draw[blue, bend right] (n4) to (v4);
            \draw[blue, bend right] (n4) to (v1);
            \draw[blue, bend right] (n3) to (v4);
            \draw[blue, bend right] (n3) to (v3);
        \end{tikzpicture}
    \end{figure}
\end{frame}

\begin{frame}
    \begin{theorem}
        \renewcommand{\qedsymbol}{}
        Zbiory $V_\emptyset \cup V_{i,i+2}$ dla każdego $i$ są niezależne.
    \end{theorem}
    \textbf{Dowód przez sprzeczność:} $x,y \in V_\emptyset \cup V_{1,3}$
    \begin{figure}[h]
        \centering
        \begin{tikzpicture}[scale=0.5, every node/.style={draw, circle, inner sep=1pt, minimum size=6pt}]
    
            % Initial cycle C_5 - pentagon shape
            \node (v1) at (90:1.5) {$v_1$};
            \node (v2) at (162:1.5) {$v_2$};
            \node (v3) at (234:1.5) {$v_3$};
            \node (v4)[fill=red] at (306:1.5) {$v_4$};
            \node (v5)[fill=red] at (18:1.5) {$v_5$};
            
            % Draw edges for the initial cycle C_5
            \draw (v1) -- (v2);
            \draw (v2) -- (v3);
            \draw (v3) -- (v4);
            \draw[red, very thick] (v4) -- (v5);
            \draw (v5) -- (v1);
        
            % First neighborhood
            \node (n2) [fill=red, ellipse, draw=red, minimum size=10pt, text=white] at ($(v1) + (2, 1)$) {$V_{1,3}$};
            \node (n3) [fill=red, ellipse, draw=red, minimum size=10pt, text=white] at ($(v1) + (-2, 1)$) {$V_{\emptyset}$};

            
            % Edges from first neighborhood to initial cycle

            \draw[black, bend right] (n2) to (v1);
            \draw[black, bend right] (n2) to (v3);
            \draw[red, very thick] (n2) to (n3);
         
        \end{tikzpicture}
    \end{figure}
    \begin{figure}[h]
        \centering
        \begin{tikzpicture}[scale=0.5, every node/.style={draw, circle, inner sep=1pt, minimum size=6pt}]
    
            % Initial cycle C_5 - pentagon shape
            \node (v1) at (90:1.5) {$v_1$};
            \node (v2) at (162:1.5) {$v_2$};
            \node (v3) at (234:1.5) {$v_3$};
            \node (v4) at (306:1.5) {$v_4$};
            \node (v5) at (18:1.5) {$v_5$};
            
            % Draw edges for the initial cycle C_5
            \draw (v1) -- (v2);
            \draw (v2) -- (v3);
            \draw (v3) -- (v4);
            \draw (v4) -- (v5);
            \draw (v5) -- (v1);
        
            % First neighborhood
            \node (n1) [fill=blue, ellipse, draw=blue, minimum size=10pt, text=white] at ($(v1) + (4, 1)$) {$V_{1,2,4}$};
            \node (n2) [fill=blue, ellipse, draw=blue, minimum size=10pt, text=white] at ($(v1) + (2, 1)$) {$V_{2,3,5}$};
            \node (n3) [fill=blue, ellipse, draw=blue, minimum size=10pt, text=white] at ($(v1) + (-2, 1)$) {$V_{1,2,3,4}$};
            \node (n4) [fill=blue, ellipse, draw=blue, minimum size=10pt, text=white] at ($(v1) + (-4, 0)$) {$V_{1,4}$};
            \node (n5) [fill=blue, ellipse, draw=blue, minimum size=10pt, text=white] at ($(v1) + (-6, 1)$) {$V_{\emptyset}$};
            \node (n6) [fill=blue, ellipse, draw=blue, minimum size=10pt, text=white] at ($(v1) + (8, 1)$) {$V_{1,2,3,4,5}$};

            % Edges from first neighborhood to initial cycle
            \draw[blue, bend left] (n6) to (v1);
            \draw[blue, bend left] (n6) to (v2);
            \draw[blue, bend left] (n6) to (v3);
            \draw[blue, bend left] (n6) to (v4);
            \draw[blue, bend left] (n6) to (v5);
            \draw[blue, bend left] (n1) to (v1);
            \draw[blue, bend left] (n1) to (v2);
            \draw[blue, bend left] (n1) to (v4);
            \draw[blue, bend right] (n4) to (v1);
            \draw[blue, bend left] (n2) to (v2);
            \draw[blue, bend left] (n2) to (v3);
            \draw[blue, bend left] (n2) to (v5);
            \draw[blue, bend right] (n3) to (v1);
            \draw[blue, bend right] (n3) to (v2);
            \draw[blue, bend right] (n3) to (v3);
            \draw[blue, bend right] (n4) to (v4);
            \draw[blue, bend right] (n3) to (v4);
            \draw[blue] (n5) to (n3);
        \end{tikzpicture}
    \end{figure}
\end{frame}

\begin{frame}
    \begin{theorem}
        \renewcommand{\qedsymbol}{}
        Dla każdego $i$, $|V_{i,i+1,i+3} \cup V_{i,i+1,i+2,i+3}| \leq 1$.
    \end{theorem}
    \textbf{Dowód przez sprzeczność:} $x,y \in V_{1,2,4} \cup V_{1,2,3,4}$
    \begin{figure}[h]
        \centering
        \begin{tikzpicture}[scale=0.5, every node/.style={draw, circle, inner sep=1pt, minimum size=6pt}]
    
            % Initial cycle C_5 - pentagon shape
            \node (v1)[fill=red] at (90:1.5) {$v_1$};
            \node (v2)[fill=red] at (162:1.5) {$v_2$};
            \node (v3) at (234:1.5) {$v_3$};
            \node (v4)[fill=red] at (306:1.5) {$v_4$};
            \node (v5) at (18:1.5) {$v_5$};
            
            % Draw edges for the initial cycle C_5
            \draw[red, very thick] (v1) -- (v2);
            \draw (v2) -- (v3);
            \draw (v3) -- (v4);
            \draw (v4) -- (v5);
            \draw (v5) -- (v1);
        
            % First neighborhood
            \node (n2) [fill=red, ellipse, draw=red, minimum size=10pt, text=white] at ($(v1) + (2, 1)$) {$V_{1,2,4}$};
            \node (n3) [fill=red, ellipse, draw=red, minimum size=10pt, text=white] at ($(v1) + (-2, 1)$) {$V_{1,2,3,4}$};

            
            % Edges from first neighborhood to initial cycle

            \draw[red, bend right, very thick] (n2) to (v1);
            \draw[red, bend right, very thick] (n2) to (v2);
            \draw[red, bend right, very thick] (n2) to (v4);
            \draw[red, bend right, very thick] (n3) to (v1);
            \draw[red, bend right, very thick] (n3) to (v2);
            \draw[black, bend right] (n3) to (v3);
            \draw[red, bend right, very thick] (n3) to (v4);
         
        \end{tikzpicture}
    \end{figure}
    \begin{figure}[h]
        \centering
        \begin{tikzpicture}[scale=0.5, every node/.style={draw, circle, inner sep=1pt, minimum size=6pt}]
    
            % Initial cycle C_5 - pentagon shape
            \node (v1) at (90:1.5) {$v_1$};
            \node (v2) at (162:1.5) {$v_2$};
            \node (v3) at (234:1.5) {$v_3$};
            \node (v4) at (306:1.5) {$v_4$};
            \node (v5) at (18:1.5) {$v_5$};
            
            % Draw edges for the initial cycle C_5
            \draw (v1) -- (v2);
            \draw (v2) -- (v3);
            \draw (v3) -- (v4);
            \draw (v4) -- (v5);
            \draw (v5) -- (v1);
        
            % First neighborhood
            \node (n2) [fill=blue, ellipse, draw=blue, minimum size=10pt, text=white] at ($(v1) + (2, 1)$) {$V_{2,3,5}$};
            \node (n3) [fill=blue, ellipse, draw=blue, minimum size=10pt, text=white] at ($(v1) + (-2, 1)$) {$V_{1,2,3,4}$};
            \node (n4) [fill=blue, ellipse, draw=blue, minimum size=10pt, text=white] at ($(v1) + (-4, 0)$) {$V_{1,4}$};
            \node (n5) [fill=blue, ellipse, draw=blue, minimum size=10pt, text=white] at ($(v1) + (-6, 1)$) {$V_{\emptyset}$};
            \node (n6) [fill=blue, ellipse, draw=blue, minimum size=10pt, text=white] at ($(v1) + (8, 1)$) {$V_{1,2,3,4,5}$};

            % Edges from first neighborhood to initial cycle
            \draw[blue, bend left] (n6) to (v1);
            \draw[blue, bend left] (n6) to (v2);
            \draw[blue, bend left] (n6) to (v3);
            \draw[blue, bend left] (n6) to (v4);
            \draw[blue, bend left] (n6) to (v5);
            \draw[blue, bend right] (n4) to (v1);
            \draw[blue, bend left] (n2) to (v2);
            \draw[blue, bend left] (n2) to (v3);
            \draw[blue, bend left] (n2) to (v5);
            \draw[blue, bend right] (n3) to (v1);
            \draw[blue, bend right] (n3) to (v2);
            \draw[blue, bend right] (n4) to (v4);
            \draw[blue, bend right] (n3) to (v3);
            \draw[blue, bend right] (n3) to (v4);
            \draw[blue] (n5) to (n3);
        \end{tikzpicture}
    \end{figure}
\end{frame}

\begin{frame}
    \begin{theorem}
        \renewcommand{\qedsymbol}{}
        Dla każdego $i$ istnieje co najwyżej jedną krawędź między $V_{i,i+2}$ a $V_{i,i-2}$.
    \end{theorem}
    \textbf{Dowód przez sprzeczność:} $x \in V_{1,3}$ i ma dwóch sąsiadów $y,y' \in V_{1,4}$
    \begin{figure}[h]
        \centering
        \begin{tikzpicture}[scale=0.5, every node/.style={draw, circle, inner sep=1pt, minimum size=6pt}]
    
            % Initial cycle C_5 - pentagon shape
            \node (v1)[fill=red] at (90:1.5) {$v_1$};
            \node (v2) at (162:1.5) {$v_2$};
            \node (v3) at (234:1.5) {$v_3$};
            \node (v4)[fill=red] at (306:1.5) {$v_4$};
            \node (v5) at (18:1.5) {$v_5$};
            
            % Draw edges for the initial cycle C_5
            \draw (v1) -- (v2);
            \draw (v2) -- (v3);
            \draw (v3) -- (v4);
            \draw (v4) -- (v5);
            \draw (v5) -- (v1);
        
            % First neighborhood
            \node (n2) [fill=red, ellipse, draw=red, minimum size=10pt, text=white] at ($(v1) + (2, 1)$) {$V_{1,3}$};
            \node (n3) [fill=red, ellipse, draw=red, minimum size=10pt, text=white] at ($(v1) + (-2, 1)$) {$V_{1,4}$};
            \node (ny) [fill=black, circle] at ($(v1) + (-1.5, 1)$) {};
            \node (nyy) [fill=black, circle] at ($(v1) + (-2.5, 1)$) {};
            \node (x) [fill=black, circle] at ($(v1) + (2, 1)$) {};

            
            % Edges from first neighborhood to initial cycle

            \draw[red, bend right, very thick] (nyy) to (v1);
            \draw[red, bend right, very thick] (nyy) to (v4);
            \draw[red, bend right, very thick] (ny) to (v4);
            \draw[red, bend right, very thick] (ny) to (x);
            \draw[red, bend right, very thick] (nyy) to (x);
            \draw[red, bend right, very thick] (v1) to (x);
            \draw[black, bend right] (v3) to (x);
         
        \end{tikzpicture}
    \end{figure}
    \begin{figure}[h]
        \centering
        \begin{tikzpicture}[scale=0.5, every node/.style={draw, circle, inner sep=1pt, minimum size=6pt}]
    
            % Initial cycle C_5 - pentagon shape
            \node (v1) at (90:1.5) {$v_1$};
            \node (v2) at (162:1.5) {$v_2$};
            \node (v3) at (234:1.5) {$v_3$};
            \node (v4) at (306:1.5) {$v_4$};
            \node (v5) at (18:1.5) {$v_5$};
            
            % Draw edges for the initial cycle C_5
            \draw (v1) -- (v2);
            \draw (v2) -- (v3);
            \draw (v3) -- (v4);
            \draw (v4) -- (v5);
            \draw (v5) -- (v1);
        
            % First neighborhood
            \node (n2) [fill=blue, ellipse, draw=blue, minimum size=10pt, text=white] at ($(v1) + (2, 1)$) {$V_{2,3,5}$};
            \node (n3) [fill=blue, ellipse, draw=blue, minimum size=10pt, text=white] at ($(v1) + (-2, 1)$) {$V_{1,2,3,4}$};
            \node (n4) [fill=blue, ellipse, draw=blue, minimum size=10pt, text=white] at ($(v1) + (-4, 0)$) {$V_{1,4}$};
            \node (n5) [fill=blue, ellipse, draw=blue, minimum size=10pt, text=white] at ($(v1) + (-6, 1)$) {$V_{\emptyset}$};
            \node (n6) [fill=blue, ellipse, draw=blue, minimum size=10pt, text=white] at ($(v1) + (8, 1)$) {$V_{1,2,3,4,5}$};
            \node (n7) [fill=blue, ellipse, draw=blue, minimum size=10pt, text=white] at ($(v1) + (-4, -2)$) {$V_{1,3}$};

            % Edges from first neighborhood to initial cycle
            \draw[blue, bend left] (n6) to (v1);
            \draw[blue, bend left] (n6) to (v2);
            \draw[blue, bend left] (n6) to (v3);
            \draw[blue, bend left] (n6) to (v4);
            \draw[blue, bend left] (n6) to (v5);
            \draw[blue, bend right] (n4) to (v1);
            \draw[blue, bend left] (n2) to (v2);
            \draw[blue, bend left] (n2) to (v3);
            \draw[blue, bend left] (n2) to (v5);
            \draw[blue, bend right] (n3) to (v1);
            \draw[blue, bend right] (n3) to (v2);
            \draw[blue, bend right] (n4) to (v4);
            \draw[blue, bend right] (n3) to (v3);
            \draw[blue, bend right] (n3) to (v4);
            \draw[blue] (n5) to (n3);
            \draw[blue] (n7) to (n4);
            \draw[blue, bend right] (n7) to (v1);
            \draw[blue, bend right] (n7) to (v3);
        \end{tikzpicture}
    \end{figure}
\end{frame}

\begin{frame}
    \begin{theorem}
        \renewcommand{\qedsymbol}{}
        Dla każdego $i$, $V_{i,i+2}$ jest pełne w stosunku do $V_{i-1,i+1} \cup V_{i+1,i+3}$.
    \end{theorem}
    \textbf{Dowód przez sprzeczność:} $x \in V_{1,3}$ i $y \in V_{2,4}$
    \begin{figure}[h]
        \centering
        \begin{tikzpicture}[scale=0.5, every node/.style={draw, circle, inner sep=1pt, minimum size=6pt}]
    
            % Initial cycle C_5 - pentagon shape
            \node (v1)[fill=red] at (90:1.5) {$v_1$};
            \node (v2) at (162:1.5) {$v_2$};
            \node (v3) at (234:1.5) {$v_3$};
            \node (v4)[fill=red] at (306:1.5) {$v_4$};
            \node (v5) at (18:1.5) {$v_5$};
            
            % Draw edges for the initial cycle C_5
            \draw (v1) -- (v2);
            \draw (v2) -- (v3);
            \draw (v3) -- (v4);
            \draw (v4) -- (v5);
            \draw (v5) -- (v1);
        
            % First neighborhood
            \node (n2) [fill=red, ellipse, draw=red, minimum size=10pt, text=white] at ($(v1) + (2, 1)$) {$V_{1,3}$};
            \node (n3) [fill=red, ellipse, draw=red, minimum size=10pt, text=white] at ($(v1) + (-2, 1)$) {$V_{2,4}$};

            
            % Edges from first neighborhood to initial cycle

            \draw[red, bend right, very thick] (n2) to (v1);
            \draw[black, bend right] (n2) to (v3);
            \draw[red, bend right, very thick] (n3) to (v4);
            \draw[black, bend right] (n3) to (v2);

         
        \end{tikzpicture}
    \end{figure}
    \begin{figure}[h]
        \centering
        \begin{tikzpicture}[scale=0.5, every node/.style={draw, circle, inner sep=1pt, minimum size=6pt}]
    
            % Initial cycle C_5 - pentagon shape
            \node (v1) at (90:1.5) {$v_1$};
            \node (v2) at (162:1.5) {$v_2$};
            \node (v3) at (234:1.5) {$v_3$};
            \node (v4) at (306:1.5) {$v_4$};
            \node (v5) at (18:1.5) {$v_5$};
            
            % Draw edges for the initial cycle C_5
            \draw (v1) -- (v2);
            \draw (v2) -- (v3);
            \draw (v3) -- (v4);
            \draw (v4) -- (v5);
            \draw (v5) -- (v1);
        
            % First neighborhood
            \node (n2) [fill=blue, ellipse, draw=blue, minimum size=10pt, text=white] at ($(v1) + (2, 1)$) {$V_{2,3,5}$};
            \node (n3) [fill=blue, ellipse, draw=blue, minimum size=10pt, text=white] at ($(v1) + (-2, 1)$) {$V_{1,2,3,4}$};
            \node (n4) [fill=blue, ellipse, draw=blue, minimum size=10pt, text=white] at ($(v1) + (-4, 0)$) {$V_{1,4}$};
            \node (n5) [fill=blue, ellipse, draw=blue, minimum size=10pt, text=white] at ($(v1) + (-6, 1)$) {$V_{\emptyset}$};
            \node (n6) [fill=blue, ellipse, draw=blue, minimum size=10pt, text=white] at ($(v1) + (8, 1)$) {$V_{1,2,3,4,5}$};
            \node (n7) [fill=blue, ellipse, draw=blue, minimum size=10pt, text=white] at ($(v1) + (-4, -2)$) {$V_{1,3}$};
            \node (n8) [fill=blue, ellipse, draw=blue, minimum size=10pt, text=white] at ($(v1) + (-6, -4)$) {$V_{2,4}$};

            % Edges from first neighborhood to initial cycle
            \draw[blue, bend left] (n6) to (v1);
            \draw[blue, bend left] (n6) to (v2);
            \draw[blue, bend left] (n6) to (v3);
            \draw[blue, bend left] (n6) to (v4);
            \draw[blue, bend left] (n6) to (v5);
            \draw[blue, bend right] (n4) to (v1);
            \draw[blue, bend left] (n2) to (v2);
            \draw[blue, bend left] (n2) to (v3);
            \draw[blue, bend left] (n2) to (v5);
            \draw[blue, bend right] (n3) to (v1);
            \draw[blue, bend right] (n3) to (v2);
            \draw[blue, bend right] (n4) to (v4);
            \draw[blue, bend right] (n3) to (v3);
            \draw[blue, bend right] (n3) to (v4);
            \draw[blue] (n5) to (n3);
            \draw[blue] (n7) to (n4);
            \draw[blue] (n7) to (n8);
            \draw[blue, bend right] (n7) to (v1);
            \draw[blue, bend right] (n7) to (v3);
            \draw[blue] (n8) to (n4);
            \draw[blue, bend right] (n8) to (v2);
            \draw[blue, bend right] (n8) to (v4);
        \end{tikzpicture}
    \end{figure}
\end{frame}

\begin{frame}
    \begin{theorem}
        \renewcommand{\qedsymbol}{}
        Jeżeli $x \in V_{1,2,3,4,5}$, to $x$ jest pełne w stosunku do $V(G) \setminus \{x\}$. W szczególności, $V_{1,2,3,4,5}$ jest kliką.
    \end{theorem}
    \begin{figure}[h]
        \centering
        \begin{tikzpicture}[scale=.75, every node/.style={draw, circle, inner sep=1pt, minimum size=6pt}]
    
            % Initial cycle C_5 - pentagon shape
            \node (v1) at (90:1.5) {$v_1$};
            \node (v2) at (162:1.5) {$v_2$};
            \node (v3) at (234:1.5) {$v_3$};
            \node (v4) at (306:1.5) {$v_4$};
            \node (v5) at (18:1.5) {$v_5$};
            
            % Draw edges for the initial cycle C_5
            \draw (v1) -- (v2);
            \draw (v2) -- (v3);
            \draw (v3) -- (v4);
            \draw (v4) -- (v5);
            \draw (v5) -- (v1);
                 
            % First neighborhood
            \node (n2) [fill=blue, ellipse, draw=blue, minimum size=10pt, text=white] at ($(v1) + (2, 1)$) {$V_{2,3,5}$};
            \node (n3) [fill=blue, ellipse, draw=blue, minimum size=10pt, text=white] at ($(v1) + (-2, 1)$) {$V_{1,2,3,4}$};
            \node (n4) [fill=blue, ellipse, draw=blue, minimum size=10pt, text=white] at ($(v1) + (-4, 0)$) {$V_{1,4}$};
            \node (n5) [fill=blue, ellipse, draw=blue, minimum size=10pt, text=white] at ($(v1) + (-6, 1)$) {$V_{\emptyset}$};
            \node (n6) [fill=blue, ellipse, draw=blue, minimum size=10pt, text=white] at ($(v1) + (8, 1)$) {$V_{1,2,3,4,5}$};
            \node (n7) [fill=blue, ellipse, draw=blue, minimum size=10pt, text=white] at ($(v1) + (-4, -2)$) {$V_{1,3}$};
            \node (n8) [fill=blue, ellipse, draw=blue, minimum size=10pt, text=white] at ($(v1) + (-6, -4)$) {$V_{2,4}$};

            % Edges from first neighborhood to initial cycle
            \draw[blue, bend left] (n6) to (v1);
            \draw[blue, bend left] (n6) to (v2);
            \draw[blue, bend left] (n6) to (v3);
            \draw[blue, bend left] (n6) to (v4);
            \draw[blue, bend left] (n6) to (v5);
            \draw[blue, bend right] (n4) to (v1);
            \draw[blue, bend left] (n2) to (v2);
            \draw[blue, bend left] (n2) to (v3);
            \draw[blue, bend left] (n2) to (v5);
            \draw[blue, bend right] (n3) to (v1);
            \draw[blue, bend right] (n3) to (v2);
            \draw[blue, bend right] (n4) to (v4);
            \draw[blue, bend right] (n3) to (v3);
            \draw[blue, bend right] (n3) to (v4);
            \draw[blue] (n5) to (n3);
            \draw[blue, dashed] (n7) to (n4);
            \draw[blue, bend right] (n7) to (v1);
            \draw[blue, bend right] (n7) to (v3);
            \draw[blue, dashed] (n8) to (n4);
            \draw[blue, bend right] (n8) to (v2);
            \draw[blue, bend right] (n8) to (v4);
            \draw[purple, bend right] (n6) to (n2);
            \draw[purple, bend right] (n6) to (n3);
            \draw[purple, bend right] (n6) to (n4);
            \draw[purple, bend right] (n6) to (n7);
            \draw[purple, bend right] (n6) to (n8);
            \draw[purple, bend right] (n6) to (n5);
            \draw[blue] (n7) to (n8);
        \end{tikzpicture}
    \end{figure}
\end{frame}

\begin{frame}
    \textbf{1.} Dla każdego $i$, $|V_{i,i+1,i+3} \cup V_{i,i+1,i+2,i+3}| \leq 1$. - usuwamy co najwyżej 5 wierzchołków.
    \begin{figure}[h]
        \centering
        \begin{tikzpicture}[scale=.75, every node/.style={draw, circle, inner sep=1pt, minimum size=6pt}]
    
            % Initial cycle C_5 - pentagon shape
            \node (v1) at (90:1.5) {$v_1$};
            \node (v2) at (162:1.5) {$v_2$};
            \node (v3) at (234:1.5) {$v_3$};
            \node (v4) at (306:1.5) {$v_4$};
            \node (v5) at (18:1.5) {$v_5$};
            
            % Draw edges for the initial cycle C_5
            \draw (v1) -- (v2);
            \draw (v2) -- (v3);
            \draw (v3) -- (v4);
            \draw (v4) -- (v5);
            \draw (v5) -- (v1);
                 
            % First neighborhood
            \node (n4) [fill=blue, ellipse, draw=blue, minimum size=10pt, text=white] at ($(v1) + (-4, 0)$) {$V_{1,4}$};
            \node (n5) [fill=blue, ellipse, draw=blue, minimum size=10pt, text=white] at ($(v1) + (-6, 1)$) {$V_{\emptyset}$};
            \node (n6) [fill=blue, ellipse, draw=blue, minimum size=10pt, text=white] at ($(v1) + (8, 1)$) {$V_{1,2,3,4,5}$};
            \node (n7) [fill=blue, ellipse, draw=blue, minimum size=10pt, text=white] at ($(v1) + (-4, -2)$) {$V_{1,3}$};
            \node (n8) [fill=blue, ellipse, draw=blue, minimum size=10pt, text=white] at ($(v1) + (-6, -4)$) {$V_{2,4}$};

            % Edges from first neighborhood to initial cycle
            \draw[blue, bend left] (n6) to (v1);
            \draw[blue, bend left] (n6) to (v2);
            \draw[blue, bend left] (n6) to (v3);
            \draw[blue, bend left] (n6) to (v4);
            \draw[blue, bend left] (n6) to (v5);
            \draw[blue, bend right] (n4) to (v1);
            \draw[blue, bend right] (n4) to (v4);
            \draw[blue, dashed] (n7) to (n4);
            \draw[blue, bend right] (n7) to (v1);
            \draw[blue, dashed] (n8) to (n4);
            \draw[blue, bend right] (n8) to (v2);
            \draw[blue, bend right] (n8) to (v4);
            \draw[purple, bend right] (n6) to (n4);
            \draw[purple, bend right] (n6) to (n7);
            \draw[purple, bend right] (n6) to (n8);
            \draw[purple, bend right] (n6) to (n5);
            \draw[blue, bend right] (n7) to (v3);
            \draw[blue] (n7) to (n8);

        \end{tikzpicture}
    \end{figure}
\end{frame}

\begin{frame}
    \textbf{2.} Jeżeli $x \in V_{1,2,3,4,5}$, to $x$ jest pełne w stosunku do $V(G) \setminus \{x\}$. W szczególności, $V_{1,2,3,4,5}$ jest kliką. - dopełnienie dwudzielne na $V_{1,2,3,4,5}$ względem reszty grafu.\\
    \textbf{3.} Wykonujemy dopełnienie na $V_{1,2,3,4,5}$ - staje się on zbiorem niezależnym.
    \begin{figure}[h]
        \centering
        \begin{tikzpicture}[scale=.75, every node/.style={draw, circle, inner sep=1pt, minimum size=6pt}]
    
            % Initial cycle C_5 - pentagon shape
            \node (v1) at (90:1.5) {$v_1$};
            \node (v2) at (162:1.5) {$v_2$};
            \node (v3) at (234:1.5) {$v_3$};
            \node (v4) at (306:1.5) {$v_4$};
            \node (v5) at (18:1.5) {$v_5$};
            
            % Draw edges for the initial cycle C_5
            \draw (v1) -- (v2);
            \draw (v2) -- (v3);
            \draw (v3) -- (v4);
            \draw (v4) -- (v5);
            \draw (v5) -- (v1);
                 
            % First neighborhood
            \node (n4) [fill=blue, ellipse, draw=blue, minimum size=10pt, text=white] at ($(v1) + (-4, 0)$) {$V_{1,4}$};
            \node (n5) [fill=blue, ellipse, draw=blue, minimum size=10pt, text=white] at ($(v1) + (-6, 1)$) {$V_{\emptyset}$};
            \node (n6) [fill=blue, ellipse, draw=blue, minimum size=10pt, text=white] at ($(v1) + (8, 1)$) {$V_{1,2,3,4,5}$};
            \node (n7) [fill=blue, ellipse, draw=blue, minimum size=10pt, text=white] at ($(v1) + (-4, -2)$) {$V_{1,3}$};
            \node (n8) [fill=blue, ellipse, draw=blue, minimum size=10pt, text=white] at ($(v1) + (-6, -4)$) {$V_{2,4}$};

            % Edges from first neighborhood to initial cycle
            \draw[blue, bend right] (n4) to (v1);
            \draw[blue, bend right] (n4) to (v4);
            \draw[blue, dashed] (n7) to (n4);
            \draw[blue, bend right] (n7) to (v1);
            \draw[blue, dashed] (n8) to (n4);
            \draw[blue, bend right] (n8) to (v2);
            \draw[blue, bend right] (n8) to (v4);
            \draw[blue, bend right] (n7) to (v3);
            \draw[blue] (n7) to (n8);

        \end{tikzpicture}
    \end{figure}
\end{frame}

\begin{frame}
    \textbf{4.} Dla każdego $i$ istnieje co najwyżej jedna krawędź między $V_{i,i+2}$ a $V_{i,i-2}$. - usuwamy co najwyżej 5 wierzchołków.\\
    Pozostałe zbiory $V$ są niezależne.
    \begin{figure}[h]
        \centering
        \begin{tikzpicture}[scale=.75, every node/.style={draw, circle, inner sep=1pt, minimum size=6pt}]
    
            % Initial cycle C_5 - pentagon shape
            \node (v1) at (90:1.5) {$v_1$};
            \node (v2) at (162:1.5) {$v_2$};
            \node (v3) at (234:1.5) {$v_3$};
            \node (v4) at (306:1.5) {$v_4$};
            \node (v5) at (18:1.5) {$v_5$};
            
            % Draw edges for the initial cycle C_5
            \draw (v1) -- (v2);
            \draw (v2) -- (v3);
            \draw (v3) -- (v4);
            \draw (v4) -- (v5);
            \draw (v5) -- (v1);
                 
            % First neighborhood
            \node (n4) [fill=blue, ellipse, draw=blue, minimum size=10pt, text=white] at ($(v1) + (-4, 0)$) {$V_{1,4}$};
            \node (n5) [fill=blue, ellipse, draw=blue, minimum size=10pt, text=white] at ($(v1) + (-6, 1)$) {$V_{\emptyset}$};
            \node (n6) [fill=blue, ellipse, draw=blue, minimum size=10pt, text=white] at ($(v1) + (8, 1)$) {$V_{1,2,3,4,5}$};
            \node (n7) [fill=blue, ellipse, draw=blue, minimum size=10pt, text=white] at ($(v1) + (-4, -2)$) {$V_{1,3}$};
            \node (n8) [fill=blue, ellipse, draw=blue, minimum size=10pt, text=white] at ($(v1) + (-6, -4)$) {$V_{2,4}$};

            % Edges from first neighborhood to initial cycle
            \draw[blue, bend right] (n4) to (v1);
            \draw[blue, bend right] (n4) to (v4);
            \draw[blue, bend right] (n7) to (v1);
            \draw[blue, bend right] (n8) to (v2);
            \draw[blue, bend right] (n8) to (v4);
            \draw[blue, bend right] (n7) to (v3);
            \draw[blue] (n7) to (n8);

        \end{tikzpicture}
    \end{figure}
\end{frame}

\begin{frame}

    \textbf{5.} Dla każdego $i$, $V_{i,i+2}$ jest pełne w stosunku do $V_{i-1,i+1} \cup V_{i+1,i+3}$. - wykonujemy 
    5 dopełnień dwudzielnych. Między parami $V_{i-1, i+1} \cup \{v_i\}$, a $V_{i, i+2} \cup \{v_{i+1}\}$
    \begin{figure}[h]
        \centering
        \begin{tikzpicture}[scale=.75, every node/.style={draw, circle, inner sep=1pt, minimum size=6pt}]
    
            % Initial cycle C_5 - pentagon shape
            \node (v1) at (90:1.5) {$v_1$};
            \node (v2) at (162:1.5) {$v_2$};
            \node (v3) at (234:1.5) {$v_3$};
            \node (v4) at (306:1.5) {$v_4$};
            \node (v5) at (18:1.5) {$v_5$};
                 
            % First neighborhood
            \node (n4) [fill=blue, ellipse, draw=blue, minimum size=10pt, text=white] at ($(v1) + (-4, 0)$) {$V_{1,4}$};
            \node (n5) [fill=blue, ellipse, draw=blue, minimum size=10pt, text=white] at ($(v1) + (-6, 1)$) {$V_{\emptyset}$};
            \node (n6) [fill=blue, ellipse, draw=blue, minimum size=10pt, text=white] at ($(v1) + (8, 1)$) {$V_{1,2,3,4,5}$};
            \node (n7) [fill=blue, ellipse, draw=blue, minimum size=10pt, text=white] at ($(v1) + (-4, -2)$) {$V_{1,3}$};
            \node (n8) [fill=blue, ellipse, draw=blue, minimum size=10pt, text=white] at ($(v1) + (-6, -4)$) {$V_{2,4}$};

            % Edges from first neighborhood to initial cycle

        \end{tikzpicture}
    \end{figure}
\end{frame}

\begin{frame}
    \begin{proof}      
        Z powyższych twierdzeń wynika, że odpowiednia liczba operacji na $G$ (10 usunięć wierzchołków, 1 dopełnienia podgrafu, 6 dopełnień dwudzielnych) prowadzi do grafu, który ma ograniczoną szerokość kliki.
        \end{proof}
\end{frame}

\begin{frame}
    \begin{lemma}
        Klasa atomów $(2P_2, \overline{P_2 + P_3})$-wolnych atomów z indukowanym $C_4$ ma ograniczoną szerokość kliki.
    \end{lemma}
    \begin{proof}
        \renewcommand{\qedsymbol}{}
        Załóżmy, że $G$ jest atomem wolnym od $(2P_2, P_2 + P_3)$, który zawiera indukowany cykl $C = v_1, v_2, v_3, v_4$. Na mocy wcześniejszego lematu, możemy założyć, że $G$ jest wolny od $C_5$. Dla $S \subseteq \{1, 2, 3, 4\}$, niech $V_S$ oznacza zbiór wierzchołków $x \in V(G) \setminus V(C)$, takich że $N(x) \cap V(C) = \{v_i \mid i \in S\}$.
    \end{proof}
\end{frame}

\begin{frame}
    \begin{theorem}
        \renewcommand{\qedsymbol}{}
        \begin{enumerate}
            \item Dla każdego $i$, $V_{i,i+1,i+2}$ jest pusty.
            \item Zbiór $V_\emptyset \cup V_i \cup V_{i+1} \cup V_{i,i+1}$ jest niezależny dla każdego $i$.
            \item Dla każdego $i$, zbiory $V_{i,i+1} \cup V_{i,i+2}$ i $V_{i,i+1} \cup V_{i+1,i+3}$ są niezależne.
            \item Graf $G[V_{1,2,3,4}]$ jest wolny od $P_1 + P_2$, a więc ma ograniczoną szerokość kliki.
            \item Dla $i \in \{1,2\}$, $V_{i,i+2}$ jest pełne względem $V_{1,2,3,4}$.
            \item Dla każdego $i$, albo $V_{i-1} \cup V_{i-1,i}$, albo $V_{i,i+1} \cup V_{i+1}$ jest puste.
            \item Jeżeli $x \in V_\emptyset$, to $x$ ma co najmniej dwóch sąsiadów w jednym z $V_{1,3}$ lub $V_{2,4}$ i jest antykompletny względem drugiego zbioru. Ponadto, $x$ jest pełny względem $V_{1,2,3,4}$.
            \item Dla każdego $i \in \{1, 2\}$, $|V_{i,i+1} \cup V_{i+2,i+3}| \leq 2$.
            \item Dla każdego $i \in \{1, 2, 3, 4\}$, $V_i$ jest pełne względem $V_{1,2,3,4}$, a co najwyżej jeden wierzchołek z $V_{i,i+2}$ ma sąsiadów w $V_i$.
        \end{enumerate} 
    \end{theorem}
\end{frame}

\begin{frame}
        \centering
        \begin{tikzpicture}[scale=0.75, every node/.style={draw, circle, inner sep=1pt, minimum size=6pt}]
    
            % Initial cycle C_5 - pentagon shape
            \node (v1) at (90:2) {$v_1$};
            \node (v2) at (180:2) {$v_2$};
            \node (v3) at (270:2) {$v_3$};
            \node (v4) at (360:2) {$v_4$};
            
            % Draw edges for the initial cycle C_5
            \draw[black] (v1) -- (v2);
            \draw[black] (v2) -- (v3);
            \draw[black] (v3) -- (v4);
            \draw[black] (v4) -- (v1);
        
            % First neighborhood
            \node (n1) [fill=blue, ellipse, draw=blue, minimum size=15pt, text=white] at ($(v1) + (0, 2)$) {$V_1$};
            \node (n2) [fill=blue, ellipse, draw=blue, minimum size=15pt, text=white] at ($(v2) + (-2, 0)$) {$V_2$};
            \node (n3) [fill=blue, ellipse, draw=blue, minimum size=15pt, text=white] at ($(v3) + (0, -2)$) {$V_3$};
            \node (n4) [fill=blue, ellipse, draw=blue, minimum size=15pt, text=white] at ($(v4) + (2, 0)$) {$V_4$};
            \node (n5) [fill=blue, ellipse, draw=blue, minimum size=15pt, text=white] at ($(0, 0)$) {$V_{1,2,3,4}$};

            \node (n12) [fill=blue, ellipse, draw=blue, minimum size=15pt, text=white] at ($(-2,2)$) {$V_{1,2}$};
            \node (n13) [fill=blue, ellipse, draw=blue, minimum size=15pt, text=white] at ($(-6,0)$) {$V_{1,3}$};
            \node (n14) [fill=blue, ellipse, draw=blue, minimum size=15pt, text=white] at ($(2,2)$) {$V_{1,4}$};
            \node (n24) [fill=blue, ellipse, draw=blue, minimum size=15pt, text=white] at ($(0,-6)$) {$V_{2,4}$};
            \node (n34) [fill=blue, ellipse, draw=blue, minimum size=15pt, text=white] at ($(2,-2)$) {$V_{3,4}$};
            \node (n23) [fill=blue, ellipse, draw=blue, minimum size=15pt, text=white] at ($(-2,-2)$) {$V_{2,3}$};
            \node (ne) [fill=blue, ellipse, draw=blue, minimum size=15pt, text=white] at ($(5,2)$) {$V_{\emptyset}$};

            
            % Edges from first neighborhood to initial cycle
            \draw[blue] (n1) to (v1);
            \draw[blue] (n2) to (v2);
            \draw[blue] (n3) to (v3);
            \draw[blue] (n4) to (v4);

            \draw[blue, bend left] (n1) to (n5);
            \draw[blue, bend left] (n2) to (n5);
            \draw[blue, bend left] (n3) to (n5);
            \draw[blue, bend left] (n4) to (n5);

            \draw[blue] (n5) to (v1);
            \draw[blue] (n5) to (v2);
            \draw[blue] (n5) to (v3);
            \draw[blue] (n5) to (v4);
            \draw[blue] (n5) to (ne);
            \draw[blue, bend left] (n5) to (n24);
            \draw[blue, bend left] (n5) to (n13);
            \draw[blue, bend left] (ne) to (n24);

            \draw[blue,bend right] (n12) to (v1);
            \draw[blue,bend right] (n12) to (v2);
            \draw[blue,bend right] (n14) to (v1);
            \draw[blue,bend right] (n14) to (v4);
            \draw[blue,bend right] (n34) to (v3);
            \draw[blue,bend right] (n34) to (v4);
            \draw[blue,bend right] (n23) to (v2);
            \draw[blue,bend right] (n23) to (v3);

            \draw[blue,bend right, dashed] (n13) to (n1);
            \draw[blue,bend right, dashed] (n24) to (n2);
    
        \end{tikzpicture}
\end{frame}

\begin{frame}
    \textbf{1.} Dla każdego $i$, albo $V_{i-1} \cup V_{i-1,i}$, albo $V_{i,i+1} \cup V_{i+1}$ jest puste.\\
    Dla każdego $i \in \{1, 2\}$, $|V_{i,i+1} \cup V_{i+2,i+3}| \leq 2$. - co najwyżej 2 operacje usunięcia wierzchołków. Graf na tym etapie może już nie być atomem.
    \centering
    \begin{tikzpicture}[scale=0.65, every node/.style={draw, circle, inner sep=1pt, minimum size=6pt}]

        % Initial cycle C_5 - pentagon shape
        \node (v1) at (90:2) {$v_1$};
        \node (v2) at (180:2) {$v_2$};
        \node (v3) at (270:2) {$v_3$};
        \node (v4) at (360:2) {$v_4$};
        
        % Draw edges for the initial cycle C_5
        \draw[black] (v1) -- (v2);
        \draw[black] (v2) -- (v3);
        \draw[black] (v3) -- (v4);
        \draw[black] (v4) -- (v1);
    
        % First neighborhood
        \node (n1) [fill=blue, ellipse, draw=blue, minimum size=15pt, text=white] at ($(v1) + (0, 2)$) {$V_1$};
        \node (n2) [fill=blue, ellipse, draw=blue, minimum size=15pt, text=white] at ($(v2) + (-2, 0)$) {$V_2$};
        \node (n3) [fill=blue, ellipse, draw=blue, minimum size=15pt, text=white] at ($(v3) + (0, -2)$) {$V_3$};
        \node (n4) [fill=blue, ellipse, draw=blue, minimum size=15pt, text=white] at ($(v4) + (2, 0)$) {$V_4$};
        \node (n5) [fill=blue, ellipse, draw=blue, minimum size=15pt, text=white] at ($(0, 0)$) {$V_{1,2,3,4}$};

        \node (n13) [fill=blue, ellipse, draw=blue, minimum size=15pt, text=white] at ($(-6,0)$) {$V_{1,3}$};
        \node (n24) [fill=blue, ellipse, draw=blue, minimum size=15pt, text=white] at ($(0,-6)$) {$V_{2,4}$};
        \node (ne) [fill=blue, ellipse, draw=blue, minimum size=15pt, text=white] at ($(5,2)$) {$V_{\emptyset}$};

        
        % Edges from first neighborhood to initial cycle
        \draw[blue] (n1) to (v1);
        \draw[blue] (n2) to (v2);
        \draw[blue] (n3) to (v3);
        \draw[blue] (n4) to (v4);

        \draw[blue, bend left] (n1) to (n5);
        \draw[blue, bend left] (n2) to (n5);
        \draw[blue, bend left] (n3) to (n5);
        \draw[blue, bend left] (n4) to (n5);

        \draw[blue] (n5) to (v1);
        \draw[blue] (n5) to (v2);
        \draw[blue] (n5) to (v3);
        \draw[blue] (n5) to (v4);
        \draw[blue] (n5) to (ne);
        \draw[blue, bend left] (n5) to (n24);
        \draw[blue, bend left] (n5) to (n13);
        \draw[blue, bend left] (ne) to (n24);

        \draw[blue,bend right, dashed] (n13) to (n1);
        \draw[blue,bend right, dashed] (n24) to (n2);

    \end{tikzpicture}
\end{frame}

\begin{frame}
    \textbf{2.} Usunięcie cyklu poprzez 4 operacje usunięcia wierzchołka.
    \centering
    \begin{tikzpicture}[scale=0.65, every node/.style={draw, circle, inner sep=1pt, minimum size=6pt}]
    
        % First neighborhood
        \node (n1) [fill=blue, ellipse, draw=blue, minimum size=15pt, text=white] at ($(v1) + (0, 2)$) {$V_1$};
        \node (n2) [fill=blue, ellipse, draw=blue, minimum size=15pt, text=white] at ($(v2) + (-2, 0)$) {$V_2$};
        \node (n3) [fill=blue, ellipse, draw=blue, minimum size=15pt, text=white] at ($(v3) + (0, -2)$) {$V_3$};
        \node (n4) [fill=blue, ellipse, draw=blue, minimum size=15pt, text=white] at ($(v4) + (2, 0)$) {$V_4$};
        \node (n5) [fill=blue, ellipse, draw=blue, minimum size=15pt, text=white] at ($(0, 0)$) {$V_{1,2,3,4}$};

        \node (n13) [fill=blue, ellipse, draw=blue, minimum size=15pt, text=white] at ($(-6,0)$) {$V_{1,3}$};
        \node (n24) [fill=blue, ellipse, draw=blue, minimum size=15pt, text=white] at ($(0,-6)$) {$V_{2,4}$};
        \node (ne) [fill=blue, ellipse, draw=blue, minimum size=15pt, text=white] at ($(5,2)$) {$V_{\emptyset}$};


        \draw[blue, bend left] (n1) to (n5);
        \draw[blue, bend left] (n2) to (n5);
        \draw[blue, bend left] (n3) to (n5);
        \draw[blue, bend left] (n4) to (n5);
        
        \draw[blue] (n5) to (ne);
        \draw[blue, bend left] (n5) to (n24);
        \draw[blue, bend left] (n5) to (n13);
        \draw[blue, bend left] (ne) to (n24);

        \draw[blue,bend right, dashed] (n13) to (n1);
        \draw[blue,bend right, dashed] (n24) to (n2);

    \end{tikzpicture}
\end{frame}

\begin{frame}
    \textbf{3.} Dla każdego $i \in \{1, 2, 3, 4\}$, $V_i$ jest pełne względem $V_{1,2,3,4}$, a co najwyżej jeden wierzchołek z $V_{i,i+2}$ ma sąsiadów w $V_i$. - usunięcie dwóch wierzchołków.
    \\
    \centering
    \begin{tikzpicture}[scale=0.65, every node/.style={draw, circle, inner sep=1pt, minimum size=6pt}]
    
        % First neighborhood
        \node (n1) [fill=blue, ellipse, draw=blue, minimum size=15pt, text=white] at ($(v1) + (0, 2)$) {$V_1$};
        \node (n2) [fill=blue, ellipse, draw=blue, minimum size=15pt, text=white] at ($(v2) + (-2, 0)$) {$V_2$};
        \node (n3) [fill=blue, ellipse, draw=blue, minimum size=15pt, text=white] at ($(v3) + (0, -2)$) {$V_3$};
        \node (n4) [fill=blue, ellipse, draw=blue, minimum size=15pt, text=white] at ($(v4) + (2, 0)$) {$V_4$};
        \node (n5) [fill=blue, ellipse, draw=blue, minimum size=15pt, text=white] at ($(0, 0)$) {$V_{1,2,3,4}$};

        \node (n13) [fill=blue, ellipse, draw=blue, minimum size=15pt, text=white] at ($(-6,0)$) {$V_{1,3}$};
        \node (n24) [fill=blue, ellipse, draw=blue, minimum size=15pt, text=white] at ($(0,-6)$) {$V_{2,4}$};
        \node (ne) [fill=blue, ellipse, draw=blue, minimum size=15pt, text=white] at ($(5,2)$) {$V_{\emptyset}$};


        \draw[blue, bend left] (n1) to (n5);
        \draw[blue, bend left] (n2) to (n5);
        \draw[blue, bend left] (n3) to (n5);
        \draw[blue, bend left] (n4) to (n5);
        
        \draw[blue] (n5) to (ne);
        \draw[blue, bend left] (n5) to (n24);
        \draw[blue, bend left] (n5) to (n13);
        \draw[blue, bend left] (ne) to (n24);

    \end{tikzpicture}
\end{frame}

\begin{frame}
    \textbf{4.} Dla każdego $i$, albo $V_{i-1} \cup V_{i-1,i}$, albo $V_{i,i+1} \cup V_{i+1}$ jest puste. - przyjmujemy $V_3$ i $V_4$ za puste.
    \\
    \centering
    \begin{tikzpicture}[scale=0.65, every node/.style={draw, circle, inner sep=1pt, minimum size=6pt}]
    
        % First neighborhood
        \node (n1) [fill=blue, ellipse, draw=blue, minimum size=15pt, text=white] at ($(v1) + (0, 2)$) {$V_1$};
        \node (n2) [fill=blue, ellipse, draw=blue, minimum size=15pt, text=white] at ($(v2) + (-2, 0)$) {$V_2$};
        \node (n3) [fill=blue, ellipse, draw=blue, minimum size=15pt, text=white] at ($(v3) + (0, -2)$) {$V_3$};
        \node (n4) [fill=blue, ellipse, draw=blue, minimum size=15pt, text=white] at ($(v4) + (2, 0)$) {$V_4$};
        \node (n5) [fill=blue, ellipse, draw=blue, minimum size=15pt, text=white] at ($(0, 0)$) {$V_{1,2,3,4}$};

        \node (n13) [fill=blue, ellipse, draw=blue, minimum size=15pt, text=white] at ($(-6,0)$) {$V_{1,3}$};
        \node (n24) [fill=blue, ellipse, draw=blue, minimum size=15pt, text=white] at ($(0,-6)$) {$V_{2,4}$};
        \node (ne) [fill=blue, ellipse, draw=blue, minimum size=15pt, text=white] at ($(5,2)$) {$V_{\emptyset}$};


        \draw[blue, bend left] (n1) to (n5);
        \draw[blue, bend left] (n2) to (n5);        
        \draw[blue] (n5) to (ne);

        \draw[blue, bend left] (n5) to (n24);
        \draw[blue, bend left] (n5) to (n13);
        \draw[blue, bend left] (ne) to (n24);

    \end{tikzpicture}
\end{frame}

\begin{frame}
    \textbf{5.} Dzięki twierdzeniom:
    \begin{itemize}
        \item Dla $i \in \{1,2\}$, $V_{i,i+2}$ jest pełne względem $V_{1,2,3,4}$.
        \item Jeżeli $x \in V_\emptyset$, to $x$ ma co najmniej dwóch sąsiadów w jednym z $V_{1,3}$ lub $V_{2,4}$ i jest antykompletny względem drugiego zbioru. Ponadto, $x$ jest pełny względem $V_{1,2,3,4}$.
        \item Dla każdego $i \in \{1, 2, 3, 4\}$, $V_i$ jest pełne względem $V_{1,2,3,4}$, a co najwyżej jeden wierzchołek z $V_{i,i+2}$ ma sąsiadów w $V_i$.
    \end{itemize}
    $V_{1,2,3,4}$ jest pełny względem $V_{\emptyset} \cup V_1 \cup V_2 \cup V_{1,3} \cup V_{2,4}$ - wykorzystujemy dopełnienie dwudzielne.
    \\
    \centering
    \begin{tikzpicture}[scale=0.5, every node/.style={draw, circle, inner sep=1pt, minimum size=6pt}]
    
        % First neighborhood
        \node (n1) [fill=blue, ellipse, draw=blue, minimum size=15pt, text=white] at ($(v1) + (0, 1)$) {$V_1$};
        \node (n2) [fill=blue, ellipse, draw=blue, minimum size=15pt, text=white] at ($(v2) + (-1, 0)$) {$V_2$};
        \node (n3) [fill=blue, ellipse, draw=blue, minimum size=15pt, text=white] at ($(v3) + (0, -1)$) {$V_3$};
        \node (n4) [fill=blue, ellipse, draw=blue, minimum size=15pt, text=white] at ($(v4) + (1, 0)$) {$V_4$};
        \node (n5) [fill=blue, ellipse, draw=blue, minimum size=15pt, text=white] at ($(0, 0)$) {$V_{1,2,3,4}$};

        \node (n13) [fill=blue, ellipse, draw=blue, minimum size=15pt, text=white] at ($(-6,0)$) {$V_{1,3}$};
        \node (n24) [fill=blue, ellipse, draw=blue, minimum size=15pt, text=white] at ($(0,-2)$) {$V_{2,4}$};
        \node (ne) [fill=blue, ellipse, draw=blue, minimum size=15pt, text=white] at ($(5,2)$) {$V_{\emptyset}$};

        \draw[blue, bend left] (ne) to (n24);

    \end{tikzpicture}
\end{frame}

\begin{frame}
    \begin{itemize}
        \item Graf $G[V_{1,2,3,4}]$ jest wolny od $P_1 + P_2$, a więc ma ograniczoną szerokość kliki.
        \item Jeżeli $x \in V_\emptyset$, to $x$ ma co najmniej dwóch sąsiadów w jednym z $V_{1,3}$ lub $V_{2,4}$ i jest antykompletny względem drugiego zbioru. Ponadto, $x$ jest pełny względem $V_{1,2,3,4}$.
    \end{itemize}
    Zatem $V_\emptyset \cup V_{2,4}$ jest łańcuchem dwudzielnym, który ma ograniczoną szerokość kliki.
    \\
    \centering
    \begin{tikzpicture}[scale=0.5, every node/.style={draw, circle, inner sep=1pt, minimum size=6pt}]
    
        % First neighborhood
        \node (n1) [fill=blue, ellipse, draw=blue, minimum size=15pt, text=white] at ($(v1) + (0, 1)$) {$V_1$};
        \node (n2) [fill=blue, ellipse, draw=blue, minimum size=15pt, text=white] at ($(v2) + (-1, 0)$) {$V_2$};
        \node (n3) [fill=blue, ellipse, draw=blue, minimum size=15pt, text=white] at ($(v3) + (0, -1)$) {$V_3$};
        \node (n4) [fill=blue, ellipse, draw=blue, minimum size=15pt, text=white] at ($(v4) + (1, 0)$) {$V_4$};
        \node (n5) [fill=blue, ellipse, draw=blue, minimum size=15pt, text=white] at ($(0, 0)$) {$V_{1,2,3,4}$};

        \node (n13) [fill=blue, ellipse, draw=blue, minimum size=15pt, text=white] at ($(-6,0)$) {$V_{1,3}$};
        \node (n24) [fill=blue, ellipse, draw=blue, minimum size=15pt, text=white] at ($(0,-2)$) {$V_{2,4}$};
        \node (ne) [fill=blue, ellipse, draw=blue, minimum size=15pt, text=white] at ($(5,2)$) {$V_{\emptyset}$};

        \draw[blue, bend left] (ne) to (n24);

    \end{tikzpicture}
\end{frame}

\begin{frame}
    \begin{proof}     
        Na podstawie powyższych twierdzeń, odpowiednia liczba operacji (usunięcia wierzchołków, komplementacje podgrafów i komplementacje dwudzielne) prowadzi do grafu o ograniczonej szerokości kliki.
        \end{proof}
\end{frame}

\begin{frame}
    \begin{theorem}
        Klasa atomów $(2P_2, \overline{P_2 + P_3})$-wolnych ma ograniczoną szerokość kliki (mając na uwadze ze klasa grafów $(2P_2, \overline{P_2 + P_3})$-wolnych ma nieograniczoną szerokość kliki)
    \end{theorem}
    \begin{proof}
        Klasa grafów podzielnych to klasa grafów wolnych od $(C_4, C_5, 2P_2)$. Ponieważ grafy podzielne tworzą podklasę klasy grafów wolnych od $(2P_2, \overline{P_2 + P_3})$, a grafy podzielne mają nieograniczoną szerokość kliki, wynika z tego, że grafy wolne od $(2P_2, \overline{P_2 + P_3})$ również mają nieograniczoną szerokość kliki. Przypomnijmy, że atomy podzielne są grafami pełnymi, a zatem ich szerokość kliki wynosi co najwyżej 2. Atomy wolne od $(2P_2, \overline{P_2 + P_3})$, które nie są podzielne, muszą zatem zawierać indukowany $C_4$ lub $C_5$.
    \end{proof}
\end{frame}

\begin{frame}{Pozostałe wyniki}
    \begin{enumerate}
        \item Klasa atomów w $G$ ma ograniczoną szerokość kliki, jeśli jest równoważna klasie grafów wolnych od $(H_1, H_2)$, gdzie spełniony jest jeden z następujących warunków:
        \begin{itemize}
            \item[(i)] $H_1$ lub $H_2 \subseteq_i P_4$
            \item[(ii)] $H_1 = K_s$ i $H_2 = tP_1$ dla pewnych $s, t \geq 1$
            \item[(iii)] $H_1 \subseteq_i \text{paw}$ i $H_2 \subseteq_i K_{1,3} + 3P_1, K_{1,3} + P_2, P_1 + P_2 + P_3, P_1 + P_5, P_1 + S_{1,1,2}, P_2 + P_4, P_6, S_{1,1,3}$ lub $S_{1,2,2}$
            \item[(iv)] $H_1 \subseteq_i \text{diamond}$ i $H_2 \subseteq_i P_1 + 2P_2, 3P_1 + P_2$ lub $P_2 + P_3$
            \item[(v)] $H_1 \subseteq_i \text{gem}$ i $H_2 \subseteq_i P_1 + P_4$ lub $P_5$
            \item[(vi)] $H_1 \subseteq_i K_3 + P_1$ i $H_2 \subseteq_i K_{1,3}$
            \item[(vii)] $H_1 \subseteq_i \overline{2P_1 + P_3}$ i $H_2 \subseteq_i 2P_1 + P_3$
        \end{itemize}
        
        \item Klasa atomów w $G$ ma ograniczoną szerokość kliki, jeśli $G$ jest podklasą klasy:
        \begin{itemize}
            \item[(i)] grafów wolnych od $(P_6, \overline{2P_2})$
            \item[(ii)] grafów wolnych od $(2P_2, \overline{P_2 + P_3})$
        \end{itemize}
    \end{enumerate}
\end{frame}

\begin{frame}{Pozostałe wyniki}
    \begin{enumerate}
        \item Klasa atomów w $G$ ma nieograniczoną szerokość kliki, jeśli jest równoważna klasie grafów wolnych od $(H_1, H_2)$, gdzie spełniony jest jeden z następujących warunków:
        \begin{itemize}
            \item[(i)] $H_1 \notin S$ i $H_2 \notin S$
            \item[(ii)] $H_1 \notin S$ i $H_2 \notin \overline{S}$
            \item[(iii)] $H_1 \supseteq_i K_3 + P_1$ i $H_2 \supseteq_i 4P_1$ lub $2P_2$
            \item[(iv)] $H_1 \supseteq_i \text{diamond}$ i $H_2 \supseteq_i K_{1,3}, 5P_1$ lub $P_2 + P_4$
            \item[(v)] $H_1 \supseteq_i K_3$ i $H_2 \supseteq_i 2P_1 + 2P_2, 2P_1 + P_4, 4P_1 + P_2, 3P_2$ lub $2P_3$
            \item[(vi)] $H_1 \supseteq_i K_4$ i $H_2 \supseteq_i P_1 + P_4, 3P_1 + P_2$ lub $2P_2$
            \item[(vii)] $H_1 \supseteq_i \text{gem}$ i $H_2 \supseteq_i P_1 + 2P_2$
        \end{itemize}
        
        \item Klasa atomów w $G$ ma nieograniczoną szerokość kliki, jeśli zawiera klasę grafów wolnych od $(H_1, H_2)$, gdzie spełniony jest jeden z następujących warunków:
        \begin{itemize}
            \item[(i)] $H_1 \supseteq_i \text{diamond}$ i $H_2 \supseteq_i P_1 + P_6$
            \item[(ii)] $H_1 \supseteq_i 2P_1 + P_2$ i $H_2 \supseteq_i P_6$
            \item[(iii)] $H_1 \supseteq_i \text{gem}$ i $H_2 \supseteq_i P_6$
            \item[(iv)] $H_1 \supseteq_i P_1 + 2P_2$ lub $P_6$ i $H_2 \supseteq_i \overline{P_1 + 2P_2}$ lub $\overline{P_2 + P_3}$
            \item[(v)] $H_1 \supseteq_i 2P_2$ i $H_2 \supseteq_i \overline{P_2 + P_4}, \overline{3P_2}$ lub $\overline{P_5}$
        \end{itemize}
    \end{enumerate}
\end{frame}

\begin{frame}{Problemy otwarte}
    Czy klasa atomów wolnych od $(H_1, H_2)$ ma ograniczoną szerokość kliki, jeśli:
    \begin{itemize}
        \item[(i)] $H_1 = \text{diamond}$ i $H_2 = P_6$
        \item[(ii)] $H_1 = C_4$ i $H_2 \in \{P_1 + 2P_2, P_2 + P_4, 3P_2\}$
        \item[(iii)] $H_1 = \overline{P_1 + 2P_2}$ i $H_2 \in \{2P_2, P_2 + P_3, P_5\}$
        \item[(iv)] $H_1 = \overline{P_2 + P_3}$ i $H_2 \in \{P_2 + P_3, P_5\}$
        \item[(v)] $H_1 = K_3$ i $H_2 \in \{P_1 + S_{1,1,3}, S_{1,2,3}\}^*$
        \item[(vi)] $H_1 = 3P_1$ i $H_2 = \overline{P_1 + S_{1,1,3}}^*$
        \item[(vii)] $H_1 = \text{diamond}$ i $H_2 \in \{P_1 + P_2 + P_3, P_1 + P_5\}^*$
        \item[(viii)] $H_1 = 2P_1 + P_2$ i $H_2 \in \{\overline{P_1 + P_2 + P_3}, \overline{P_1 + P_5}\}^*$
        \item[(ix)] $H_1 = \text{gem}$ i $H_2 = P_2 + P_3^*$
        \item[(x)] $H_1 = P_1 + P_4$ i $H_2 = \overline{P_2 + P_3}^*$
    \end{itemize}
    Przypadki oznaczone gwiazdką (*) oznaczają, że nie jest znana ograniczoność szerokości kliki dla całej klasy grafów wolnych od $(H_1, H_2)$ 
\end{frame}

\setbeamercovered{transparent}
\nocite{c_a_r_orig, c_s_r, b_s_p, h_p_a}
\begin{frame}[allowframebreaks]{Bibliografia}
    \bibliographystyle{plain}
    \bibliography{bibliography.bib}
\end{frame}

\begin{frame}
    \begin{center}
        {\huge Dziękuję za uwagę!}
    \end{center}
\end{frame}

%%% DOCUMENT ENDS HERE. Good bye! :) %%%

\end{document}
